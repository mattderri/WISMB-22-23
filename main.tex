%%%%%%%%%%%%%%%%%%%%%%%%%%%%%%%%%%%%%%%%%
% kaobook
% LaTeX Template
% Version 1.2 (4/1/2020)
%
% This template originates from:
% https://www.LaTeXTemplates.com
%
% For the latest template development version and to make contributions:
% https://github.com/fmarotta/kaobook
%
% Authors:
% Federico Marotta (federicomarotta@mail.com)
% Based on the doctoral thesis of Ken Arroyo Ohori (https://3d.bk.tudelft.nl/ken/en)
% and on the Tufte-LaTeX class.
% Modified for LaTeX Templates by Vel (vel@latextemplates.com)
%
% License:
% CC0 1.0 Universal (see included MANIFEST.md file)
%
%%%%%%%%%%%%%%%%%%%%%%%%%%%%%%%%%%%%%%%%%

%----------------------------------------------------------------------------------------
%	PACKAGES AND OTHER DOCUMENT CONFIGURATIONS
%----------------------------------------------------------------------------------------

\documentclass[
	fontsize=10pt, % Base font size
	twoside=false, % Use different layouts for even and odd pages (in particular, if twoside=true, the margin column will be always on the outside)
	%open=any, % If twoside=true, uncomment this to force new chapters to start on any page, not only on right (odd) pages
	%chapterprefix=true, % Uncomment to use the word "Chapter" before chapter numbers everywhere they appear
	%chapterentrydots=true, % Uncomment to output dots from the chapter name to the page number in the table of contents
	numbers=noenddot, % Comment to output dots after chapter numbers; the most common values for this option are: enddot, noenddot and auto (see the KOMAScript documentation for an in-depth explanation)
	%draft=true, % If uncommented, rulers will be added in the header and footer
	%overfullrule=true, % If uncommented, overly long lines will be marked by a black box; useful for correcting spacing problems
]{kaobook}

% Set the language
\usepackage[italian]{babel} % Load characters and hyphenation
\usepackage[english=british]{csquotes} % English quotes

% Load packages for testing
\usepackage{blindtext}
%\usepackage{showframe} % Uncomment to show boxes around the text area, margin, header and footer
%\usepackage{showlabels} % Uncomment to output the content of \label commands to the document where they are used

% Load the bibliography package
\usepackage{styles/kaobiblio}
\addbibresource{main.bib} % Bibliography file

% Load mathematical packages for theorems and related environments. NOTE: choose only one between 'mdftheorems' and 'plaintheorems'.
\usepackage{styles/mdftheorems}
%\usepackage{styles/plaintheorems}
\usepackage{physics}
\usepackage{tikz, mathtools}
\usepackage{boxedminipage}
\usepackage{svg}
\usepackage{wrapfig}
\usetikzlibrary{arrows.meta, bending, chains}
\usepackage[T1]{fontenc}
\usepackage[utf8]{inputenc}
\usepackage{calrsfs}
\usepackage{nccmath}
\usepackage{witharrows}
\DeclareMathAlphabet{\pazocal}{OMS}{zplm}{m}{n}
\usepackage{enumitem}
\usepackage{tikz-cd} % For the curved arrows under the equations
\newcommand\bigLambda{\makebox(3,3){\text{\huge $\Lambda_{\mu\nu}$}}}
\newcommand\bigR{\makebox(3,3){\text{\huge $R$}}}
\newcommand{\rvline}{\hspace*{-\arraycolsep}\vline\hspace*{-\arraycolsep}}
\usepackage{nicematrix} %Fa le matrici ciotte
\usepackage{tikz}
\usepackage{blkarray}
\usepackage{multirow}
\usepackage{arydshln}
%\usepackage[fleqn,tbtags]{mathtools}
\newcommand*\circled[1]{\tikz[baseline=(char.base)]{
            \node[shape=circle,draw,inner sep=2pt] (char) {#1};}}
% \tikzstyle{H} = [solid,<->,yshift=-2mm]
% \tikzstyle{V} = [solid,<->,xshift=8mm]
%\newcommand{\eps}{\varepsilon}
\newcommand*{\img}[1]{%
    \raisebox{-1.9\baselineskip}{%
        \includegraphics[
        height=2cm,
        width=4cm,
        keepaspectratio,
        ]{#1}%
    }%
}
\newcommand*\mycirc[1]{%
   \begin{tikzpicture}
     \node[draw,circle,inner sep=1pt] {#1};
   \end{tikzpicture}}
\graphicspath{{examples/documentation/images/}{images/}} % Paths in which to look for images
\usepackage{cancel}
\usepackage{nicefrac}
\usepackage{graphicx,calc}
\newlength\myheight
\newlength\mydepth
\settototalheight\myheight{Xygp}
\settodepth\mydepth{Xygp}
\setlength\fboxsep{0pt}
\usepackage{subfiles} % Best loaded last in the preamble
\usepackage{soul}
\usepackage{xcolor}
\definecolor{red}{RGB}{255, 0, 0}
\definecolor{blue}{RGB}{0, 0, 255}
\renewcommand{\CancelColor}{\color{red}}
\makeindex[columns=3, title=Indice analitico, intoc]% Make LaTeX produce the files required to compile the index
%~~~~~~~~~~~~~~~~~~~~
\usepackage{slashed}
\usepackage{braket}
\usepackage[super]{nth}
\usepackage{amssymb}% http://ctan.org/pkg/amssymb
\usepackage{pifont}% http://ctan.org/pkg/pifont
\newcommand{\cmark}{\ding{51}}%
\newcommand{\xmark}{\ding{55}}%

%\usepackage[compat=1.0.0]{tikz-feynman}

\makeglossaries % Make LaTeX produce the files required to compile the glossary

\makenomenclature % Make LaTeX produce the files required to compile the nomenclature

% Reset sidenote counter at chapters
%\counterwithin*{sidenote}{chapter}

%----------------------------------------------------------------------------------------

\begin{document}

%----------------------------------------------------------------------------------------
%	BOOK INFORMATION
%----------------------------------------------------------------------------------------

\titlehead{The \texttt{kaobook} class}
\subject{Appunti del corso di}

\title[Weak Interactions in the Standard Model and Beyond]{Weak Interactions in the Standard Model and Beyond}
\subtitle{Anno accademico 2022/2023\\Corso tenuto dai Proff. Roberto Contino \& Marco Nardecchia}

\author[Matteo D'Errigo]{Matteo D'Errigo\thanks{Un amante dei panini}}

\date{\today}

%\publishers{Nemo}

%----------------------------------------------------------------------------------------

\frontmatter % Denotes the start of the pre-document content, uses roman numerals

%----------------------------------------------------------------------------------------
%	OPENING PAGE
%----------------------------------------------------------------------------------------

% \makeatletter
% \extratitle{
%	% In the title page, the title is vspaced by 9.5\baselineskip
%	\vspace*{9\baselineskip}
%	\vspace*{\parskip}
%	\begin{center}
%		% In the title page, \huge is set after the komafont for title
%		\usekomafont{title}\huge\@title
%	\end{center}
%}
%\makeatother

%----------------------------------------------------------------------------------------
%	COPYRIGHT PAGE
%----------------------------------------------------------------------------------------

\makeatletter
\uppertitleback{\@titlehead} % Header

\lowertitleback{
    \textbf{Versione aggiornata}\\
	Puoi trovare la versione aggiornata di questo testo, insieme al suo codice \LaTeX  all'indirizzo:\\
	\url{https://it.overleaf.com/project/6421cb2c985d3113d7451602}\\
	%E una versione italiana del testo qua:\\
	%\url{https://www.overleaf.com/read/ntmsgsnnjjqs}\\
	
	\medskip
	
	\textbf{No copyright}\\
	\cczero\ This book is released into the public domain using the CC0 code. To the extent possible under law, I waive all copyright and related or neighbouring rights to this work.
	
	Per vedere una copia del codice cc0 si può visitare il sito: \\\url{http://creativecommons.org/publicdomain/zero/1.0/}
	
	\medskip
	
	\textbf{Colophon} \\
	This document was typeset with the help of \href{https://sourceforge.net/projects/koma-script/}{\KOMAScript} and \href{https://www.latex-project.org/}{\LaTeX} using the \href{https://github.com/fmarotta/kaobook/}{kaobook} class.
	
	Il codice sorgente di questo libro è disponibile su:\\\url{https://github.com/fmarotta/kaobook}
	
	(Il tuo contributo è benvenuto!)
	
	\medskip
	


	
	    
	\textbf{Publisher} \\
	First wrote in October 2022 by me :)
}
\makeatother

%----------------------------------------------------------------------------------------
%	DEDICATION
%----------------------------------------------------------------------------------------

% \dedication{
% 	Symmetry, as wide or narrow as you may define its meaning, is one idea by which man through the ages has tried to comprehend and create order, beauty and perfection.\\
% 	\begin{flushright} -- H. Weyl, Symmetry
% 	\end{flushright}
% 	\newline
% 	\small{\newline La simmetria, per quanto ampio o stretto si possa definire il suo significato, è un'idea con cui l'uomo nel corso dei secoli ha cercato di comprendere e creare ordine, bellezza e perfezione.}\\
% 	\begin{flushright} -- H. Weyl, Symmetry	
% 	\end{flushright}    
% }

%----------------------------------------------------------------------------------------
%	OUTPUT TITLE PAGE AND PREVIOUS
%----------------------------------------------------------------------------------------

% Note that \maketitle outputs the pages before here

% If twoside=false, \uppertitleback and \lowertitleback are not printed
% To overcome this issue, we set twoside=semi just before printing the title pages, and set it back to false just after the title pages
\KOMAoptions{twoside=semi}
\maketitle
\KOMAoptions{twoside=false}

%----------------------------------------------------------------------------------------
%	PREFACE
%----------------------------------------------------------------------------------------

%\input{chapters/preface.tex}

%----------------------------------------------------------------------------------------
%	TABLE OF CONTENTS & LIST OF FIGURES/TABLES
%----------------------------------------------------------------------------------------

\begingroup % Local scope for the following commands

% Define the style for the TOC, LOF, and LOT
%\setstretch{1} % Uncomment to modify line spacing in the ToC
%\hypersetup{linkcolor=blue} % Uncomment to set the colour of links in the ToC
\setlength{\textheight}{23cm} % Manually adjust the height of the ToC pages

% Turn on compatibility mode for the etoc package
\etocstandarddisplaystyle % "toc display" as if etoc was not loaded
\etocstandardlines % toc lines as if etoc was not loaded

\tableofcontents % Output the table of contents

%\listoffigures % Output the list of figures

% Comment both of the following lines to have the LOF and the LOT on different pages
\let\cleardoublepage\bigskip
\let\clearpage\bigskip

%\listoftables % Output the list of tables

\endgroup

%----------------------------------------------------------------------------------------
%	MAIN BODY
%----------------------------------------------------------------------------------------

\mainmatter % Denotes the start of the main document content, resets page numbering and uses arabic numbers
\setchapterstyle{kao} % Choose the default chapter heading style

%\part{Introduction}
\subfile{chapters/1_Intro}
%\part{Manifold and differential forms}
\subfile{chapters/2_Nardecchia}
\subfile{chapters/3_Contino}

% \subfile{chapters/4_differential_forms_I}
% \subfile{chapters/5_differential_forms_II}
% \subfile{chapters/6_Electrodynamics and differential forms}
% \part{Groups}
% \dedication{
% 	The beginner should not be discouraged if he finds that he does not have the prerequisites to read the prerequisites.\\
% 	\begin{flushright} -- \href{https://it.wikipedia.org/wiki/Paul_Halmos}{P. Halmos} in \cite{Halmos1950}
% 	\end{flushright}
% 	\newline
% 	\small{\newline Il principiante non deve scoraggiarsi se scopre di non avere i prerequisiti per leggere i prerequisiti.}\\
% 	\begin{flushright} -- P. Halmos
% 	\end{flushright}    
% }
% \subfile{chapters/7_Fundamental concepts}
% \subfile{chapters/8_Lie_Groups_I}
% \subfile{chapters/9_Lie_Groups_II}
% \subfile{chapters/10_Lie_Groups_III}
% \subfile{chapters/11_Two_Brothers}
% \subfile{chapters/12_Matrix_Exponential_and_Logarithm}
% \subfile{chapters/13_Tool:Polar_decomposition}
% \subfile{chapters/14_Lie_Algebra_of_a_Matrix_Lie_Group}
% \subfile{chapters/15_The exponential map_of_G}
% \subfile{chapters/16_Group_Theory_and_Quantum_Mechanics}
% \subfile{chapters/17_Representation_Theory}
% \subfile{chapters/18_Representations_of_sl2c}
% \subfile{chapters/19_Representations_of_sl3c}
% \subfile{chapters/20_Lorentz_and_Poincare_groups}
% \part{Exercises}
% \subfile{chapters/21_Starred_Exercises}

%\appendix % From here onwards, chapters are numbered with letters, as is the appendix convention

%\pagelayout{wide} % No margins
%\addpart{Appendice}
%\pagelayout{margin} % Restore margins

%\input{chapters/appendix.tex}

%----------------------------------------------------------------------------------------

\backmatter % Denotes the end of the main document content
\setchapterstyle{plain} % Output plain chapters from this point onwards
\subfile{chapters/Fish}
%----------------------------------------------------------------------------------------
%	BIBLIOGRAPHY
%----------------------------------------------------------------------------------------

% The bibliography needs to be compiled with biber using your LaTeX editor, or on the command line with 'biber main' from the template directory

\defbibnote{bibnote}{Here are listed the references sorted by citation in the text.\par\bigskip} % Prepend this text to the bibliography
\printbibliography[heading=bibintoc, title=References, prenote=bibnote] % Add the bibliography heading to the ToC, set the title of the bibliography and output the bibliography note

%----------------------------------------------------------------------------------------
%	NOMENCLATURE
%----------------------------------------------------------------------------------------

% The nomenclature needs to be compiled on the command line with 'makeindex main.nlo -s nomencl.ist -o main.nls' from the template directory

% \nomenclature{$c$}{Velocità della luce nel vuoto in un sistema di riferimento inerziale}
% \nomenclature{$h$}{Costante di Planck}
% \nomenclature{$H$}{Hamiltoniana o funzione di Hamilton}
% \nomenclature{$\mathbb{H}$}{Insieme dei quaterioni}
% \nomenclature{$\mathcal{H}$}{Spazio di Hilbert - \textit{Hilbert space}}
% \nomenclature{$\mathbb{N}$}{Numeri naturali - \textit{Natural numbers}}
% \nomenclature{$E$}{Spazio di Banach - \textit{Banach space}}
% \nomenclature{$U$}{Operatore unitario}
% \nomenclature{$\textrm{U}$}{Gruppo di Lie degli operatori unitari}
% \nomenclature{$\pazocal{U}$}{Gruppo degli operatori unitari nello spazio di Hilbert (usato in QM)}
% \nomenclature{$\mathfrak{g}, \mathfrak{h}$}{Algebra di Lie (matriciale) - \textit{(matrix) Lie algebra}}

% \renewcommand{\nomname}{Notazione} % Rename the default 'Nomenclature'
% \renewcommand{\nompreamble}{Il seguente elenco descrive alcuni simboli che sono usati in questo documento.} % Prepend this text to the nomenclature

% \printnomenclature % Output the nomenclature

% %----------------------------------------------------------------------------------------
% %	GREEK ALPHABET
% % 	Originally from https://gitlab.com/jim.hefferon/linear-algebra
% %----------------------------------------------------------------------------------------

% \vspace{1cm}

% {\usekomafont{chapter}Lettere greche con pronuncia inglese} \\[2ex]
% \begin{center}
% 	\newcommand{\pronounced}[1]{\hspace*{.2em}\small\textit{#1}}
% 	\begin{tabular}{l l @{\hspace*{3em}} l l}
% 		\toprule
% 		Carattere & Nome & Carattere & Nome \\ 
% 		\midrule
% 		$\alpha$ & alpha \pronounced{AL-fuh} & $\nu$ & nu \pronounced{NEW} \\
% 		$\beta$ & beta \pronounced{BAY-tuh} & $\xi$, $\Xi$ & xi \pronounced{KSIGH} \\ 
% 		$\gamma$, $\Gamma$ & gamma \pronounced{GAM-muh} & o & omicron \pronounced{OM-uh-CRON} \\
% 		$\delta$, $\Delta$ & delta \pronounced{DEL-tuh} & $\pi$, $\Pi$ & pi \pronounced{PIE} \\
% 		$\epsilon$ & epsilon \pronounced{EP-suh-lon} & $\rho$ & rho \pronounced{ROW} \\
% 		$\zeta$ & zeta \pronounced{ZAY-tuh} & $\sigma$, $\Sigma$ & sigma \pronounced{SIG-muh} \\
% 		$\eta$ & eta \pronounced{AY-tuh} & $\tau$ & tau \pronounced{TOW (as in cow)} \\
% 		$\theta$, $\Theta$ & theta \pronounced{THAY-tuh} & $\upsilon$, $\Upsilon$ & upsilon \pronounced{OOP-suh-LON} \\
% 		$\iota$ & iota \pronounced{eye-OH-tuh} & $\phi$, $\Phi$ & phi \pronounced{FEE, or FI (as in hi)} \\
% 		$\kappa$ & kappa \pronounced{KAP-uh} & $\chi$ & chi \pronounced{KI (as in hi)} \\
% 		$\lambda$, $\Lambda$ & lambda \pronounced{LAM-duh} & $\psi$, $\Psi$ & psi \pronounced{SIGH, or PSIGH} \\
% 		$\mu$ & mu \pronounced{MEW} & $\omega$, $\Omega$ & omega \pronounced{oh-MAY-guh} \\
% 		\bottomrule
% 	\end{tabular} \\[1.5ex]
% 	Capitals shown are the ones that differ from Roman capitals.
% \end{center}

%----------------------------------------------------------------------------------------
%	GLOSSARY
%----------------------------------------------------------------------------------------

% The glossary needs to be compiled on the command line with 'makeglossaries main' from the template directory

\newglossaryentry{computer}{
	name=computer,
	description={is a programmable machine that receives input, stores and manipulates data, and provides output in a useful format}
}

% Glossary entries (used in text with e.g. \acrfull{fpsLabel} or \acrshort{fpsLabel})
\newacronym[longplural={Frames per Second}]{fpsLabel}{FPS}{Frame per Second}
\newacronym[longplural={Tables of Contents}]{tocLabel}{TOC}{Table of Contents}

\setglossarystyle{listgroup} % Set the style of the glossary (see https://en.wikibooks.org/wiki/LaTeX/Glossary for a reference)
\printglossary[title=Special Terms, toctitle=List of Terms] % Output the glossary, 'title' is the chapter heading for the glossary, toctitle is the table of contents heading

%----------------------------------------------------------------------------------------
%	INDEX
%----------------------------------------------------------------------------------------

% The index needs to be compiled on the command line with 'makeindex main' from the template directory

\printindex % Output the index

%----------------------------------------------------------------------------------------
%	BACK COVER
%----------------------------------------------------------------------------------------

% If you have a PDF/image file that you want to use as a back cover, uncomment the following lines

%\clearpage
%\thispagestyle{empty}
%\null%
%\clearpage
%\includepdf{cover-back.pdf}

%----------------------------------------------------------------------------------------

\end{document}
